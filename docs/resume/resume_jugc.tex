%%%%%%%%%%%%%%%%%%%%%%%%%%%%%%%%%%%%%%%%%%%%%%%%%%%%%%%%%%%%%%%%%%%%%%%%
%%%%%%%%%%%%%%%%%%%%%% Simple LaTeX CV Template %%%%%%%%%%%%%%%%%%%%%%%%
%%%%%%%%%%%%%%%%%%%%%%%%%%%%%%%%%%%%%%%%%%%%%%%%%%%%%%%%%%%%%%%%%%%%%%%%

%%%%%%%%%%%%%%%%%%%%%%%%%%%%%%%%%%%%%%%%%%%%%%%%%%%%%%%%%%%%%%%%%%%%%%%%
%% NOTE: If you find that it says                                     %%
%%                                                                    %%
%%                           1 of ??                                  %%
%%                                                                    %%
%% at the bottom of your first page, this means that the AUX file     %%
%% was not available when you ran LaTeX on this source. Simply RERUN  %%
%% LaTeX to get the ``??'' replaced with the number of the last page  %%
%% of the document. The AUX file will be generated on the first run   %%
%% of LaTeX and used on the second run to fill in all of the          %%
%% references.                                                        %%
%%%%%%%%%%%%%%%%%%%%%%%%%%%%%%%%%%%%%%%%%%%%%%%%%%%%%%%%%%%%%%%%%%%%%%%%

%%%%%%%%%%%%%%%%%%%%%%%%%%%% Document Setup %%%%%%%%%%%%%%%%%%%%%%%%%%%%

% Don't like 10pt? Try 11pt or 12pt
\documentclass[10pt]{article}

% The automated optical recognition software used to digitize resume
% information works best with fonts that do not have serifs. This
% command uses a sans serif font throughout. Uncomment both lines (or at
% least the second) to restore a Roman font (i.e., a font with serifs).
%\usepackage{times}
%\renewcommand{\familydefault}{\sfdefault}

% This is a helpful package that puts math inside length specifications
\usepackage{calc}
\usepackage{comment}
\usepackage[english,danish]{babel}
%\usepackage{enumitem}

% Simpler bibsection for CV sections
% (thanks to natbib for inspiration)
\makeatletter
\newlength{\bibhang}
\setlength{\bibhang}{1em} %1em}
\newlength{\bibsep}
 {\@listi \global\bibsep\itemsep \global\advance\bibsep by\parsep}
\newenvironment{bibsection}%
        {\begin{enumerate}{}{%
%        {\begin{list}{}{%
       \setlength{\leftmargin}{\bibhang}%
       \setlength{\itemindent}{-\leftmargin}%
       \setlength{\itemsep}{\bibsep}%
       \setlength{\parsep}{\z@}%
        \setlength{\partopsep}{0pt}%
        \setlength{\topsep}{0pt}}}
        {\end{enumerate}\vspace{-.6\baselineskip}}
%        {\end{list}\vspace{-.6\baselineskip}}
\makeatother

% Layout: Puts the section titles on left side of page
\reversemarginpar



%% Use these lines for letter-sized paper
\usepackage[paper=letterpaper,
            %includefoot, % Uncomment to put page number above margin
            marginparwidth=1.2in,     % Length of section titles
            marginparsep=.05in,       % Space between titles and text
            margin=1in,               % 1 inch margins
            includemp]{geometry}


%% More layout: Get rid of indenting throughout entire document
\setlength{\parindent}{0in}

\usepackage[shortlabels]{enumitem}


%
\usepackage{fancyhdr,lastpage}
\pagestyle{fancy}
%\pagestyle{empty}      % Uncomment this to get rid of page numbers
\fancyhf{}\renewcommand{\headrulewidth}{0pt}
\fancyfootoffset{\marginparsep+\marginparwidth}
\newlength{\footpageshift}
\setlength{\footpageshift}
          {0.5\textwidth+0.5\marginparsep+0.5\marginparwidth-2in}
\lfoot{\hspace{\footpageshift}%
       \parbox{4in}{\, \hfill %
                    \arabic{page} of \protect\pageref*{LastPage} % +LP
%                    \arabic{page}                               % -LP
                    \hfill \,}}
\rhead{Juan Gallego-Calderon, Ph.D.}

% Finally, give us PDF bookmarks
\usepackage{color,hyperref}
\definecolor{darkblue}{rgb}{0.0,0.0,0.3}
\hypersetup{colorlinks,breaklinks,
            linkcolor=darkblue,urlcolor=darkblue,
            anchorcolor=darkblue,citecolor=darkblue}



\newcommand{\makeheading}[2][]%
        {\hspace*{-\marginparsep minus \marginparwidth}%
         \begin{minipage}[t]{\textwidth+\marginparwidth+\marginparsep}%
             {\large \bfseries #2 \hfill #1}\\[-0.15\baselineskip]%
                 \rule{\columnwidth}{1pt}%
         \end{minipage}}

% The section headings
%
% Usage: \section{section name}
\renewcommand{\section}[1]{\pagebreak[3]%
    \hyphenpenalty=10000%
    \vspace{1.3\baselineskip}%
    \phantomsection\addcontentsline{toc}{section}{#1}%
    \noindent\llap{\scshape\smash{\parbox[t]{\marginparwidth}{\raggedright #1}}}%
    \vspace{-\baselineskip}\par}

% An itemize-style list with lots of space between items
\newenvironment{outerlist}[1][\enskip\textbullet]%
        {\begin{itemize}[#1,leftmargin=*]}{\end{itemize}%
         \vspace{-.6\baselineskip}}
		%\vspace{0mm}}
% An environment IDENTICAL to outerlist that has better pre-list spacing
% when used as the first thing in a \section
\newenvironment{lonelist}[1][\enskip\textbullet]%
        {\begin{list}{#1}{%
        \setlength{\partopsep}{0pt}%
        \setlength{\topsep}{0pt}}}
        {\end{list}\vspace{-.6\baselineskip}}

% An itemize-style list with little space between items
\newenvironment{innerlist}[1][\enskip\textbullet]%
        {\begin{itemize}[#1,leftmargin=*,parsep=0pt,itemsep=0pt,topsep=0pt,partopsep=0pt]}
        {\end{itemize}}

% An environment IDENTICAL to innerlist that has better pre-list spacing
% when used as the first thing in a \section
\newenvironment{loneinnerlist}[1][\enskip\textbullet]%
        {\begin{itemize}[#1,leftmargin=*,parsep=0pt,itemsep=0pt,topsep=0pt,partopsep=0pt]}
        {\end{itemize}\vspace{-.6\baselineskip}}

% To add some paragraph space between lines.
% This also tells LaTeX to preferably break a page on one of these gaps
% if there is a needed pagebreak nearby.
\newcommand{\blankline}{\quad\pagebreak[3]}
\newcommand{\halfblankline}{\quad\vspace{-0.5\baselineskip}\pagebreak[3]}

% Uses hyperref to link DOI
\newcommand\doilink[1]{\href{http://dx.doi.org/#1}{#1}}
\newcommand\doi[1]{doi:\doilink{#1}}

% For \url{SOME_URL}, links SOME_URL to the url SOME_URL
\providecommand*\url[1]{\href{#1}{#1}}
% Same as above, but pretty-prints SOME_URL in teletype fixed-width font
\renewcommand*\url[1]{\href{#1}{\texttt{#1}}}

% For \email{ADDRESS}, links ADDRESS to the url mailto:ADDRESS
\providecommand*\email[1]{\href{mailto:#1}{#1}}
% Same as above, but pretty-prints ADDRESS in teletype fixed-width font
%\renewcommand*\email[1]{\href{mailto:#1}{\texttt{#1}}}

%\providecommand\BibTeX{{\rm B\kern-.05em{\sc i\kern-.025em b}\kern-.08em
%    T\kern-.1667em\lower.7ex\hbox{E}\kern-.125emX}}
%\providecommand\BibTeX{{\rm B\kern-.05em{\sc i\kern-.025em b}\kern-.08em
%    \TeX}}
\providecommand\BibTeX{{B\kern-.05em{\sc i\kern-.025em b}\kern-.08em
    \TeX}}
\providecommand\Matlab{\textsc{Matlab}}

%%%%%%%%%%%%%%%%%%%%%%%% End Helper Commands %%%%%%%%%%%%%%%%%%%%%%%%%%%

%%%%%%%%%%%%%%%%%%%%%%%%% Begin CV Document %%%%%%%%%%%%%%%%%%%%%%%%%%%%

\begin{document}\thispagestyle{empty}
\makeheading{Juan Gallego-Calderon, Ph.D.}

\section{Contact Information}

% NOTE: Mind where the & separators and \\ breaks are in the following
%       table.
%
% ALSO: \rcollength is the width of the right column of the table
%       (adjust it to your liking; default is 1.85in).
%
\newlength{\rcollength}\setlength{\rcollength}{1.4in}%
%
\begin{tabular}[t]{@{}p{\textwidth-\rcollength}p{\rcollength}}
%\href{http://www.cse.osu.edu/}%
%     {Department of Computer Science and Engineering} & \\
%\href{http://www.osu.edu/}{The Ohio State University}
Gullandsgade 8, 4tv  & +45 81941263 \\
2300 Copenhagen, Denmark     & \email{jugc@dtu.dk}\\ & \email{jfgallego2@gmail.com} \\
\end{tabular}

%\section{Objective}

%Insert text here if you want to
%\begin{innerlist}
%\item More information and auxiliary documents can be found at\\\url{http://www.tedpavlic.com/facjobsearch/}
%\end{innerlist}
\section{Professional Summary}

Over three years of research experience in the wind energy sector in the following areas:
\begin{itemize}
 \item Use of aeroelastic tool HAWC2 for computing structural loads in the 5 MW NREL reference wind turbine following the IEC 61400-1 and IEC 61400-4.
 \item Software development of numerical models to simulate the dynamics of a wind turbine drivetrain in order to estimate the internal loading in the components. The theoretical basis of these models are multibody dynamics.
 \item Use of probabilistic models in order to predict the fatigue and reliability of critical components in the wind turbine drivetrain.
 \item Use of wind turbine controller and generator controller, in order to couple HAWC2 with Matlab/Simulink to carry out Design Load Cases including a detailed drivetrain model.
\end{itemize}

\section{Theoretical background}

\begin{itemize}
 \item Multibody dynamics for model development and simulation of wind turbine structures.
 \item Control systems and system integration in wind turbines.
 \item Generator dynamics.
 \item Machinery and structural dynamics.
\end{itemize}
\section{Research Experience}

\textbf{DTU Wind Energy} \hfill {March 2015 to present} \\
\textit{Postdoc, Wind Turbine Structures section}
\begin{innerlist}
\item Modeling of drivetrain components.
\item Scripting for large quantity of batch simulations.
\item Maintain and develop further the in-house drivetrain simulation tool.
\end{innerlist}
\halfblankline

\textbf{DTU Wind Energy} \hfill {March 2012 to March 2015} \\
\textit{Ph.D. student, Wind Turbine Structures section}
\begin{innerlist}
\item Authored, developed and implemented a software capable of simulating the electromechanical drivetrain interaction.
\item Assisted in the development of the drivetrain test facilities at Risø DTU Campus.
\end{innerlist}
\halfblankline

\textbf{National Renewable Energy Laboratory (NREL)} \hfill {March 2014 to June 2014} \\
\textit{Visiting Ph.D. student, National Wind Technology Center}
\begin{innerlist}
\item Validation of drivetrain models using experimental and field data, based on load measurements.
\end{innerlist}
\halfblankline


\section{Education}
%\begin{flushleft}
\href{http://www.dtu.dk}{\textbf{Technical University of Denmark}}, Lyngby, Denmark,

        Ph.D., \href{http://www.vindenergi.dtu.dk/english/About/Sections/Wind_turbines}
             {Wind Energy},
             August 2015
        \begin{innerlist}
        \item[] Thesis Topic: \href{http://orbit.dtu.dk/en/publications/electromechanical-drivetrain-simulation\%284cc15b63-4c63-4586-9edd-4395fb121438\%29.html}{\emph{Electromechanical Drivetrain Simulation}}
        \item[] Advisors:
              \href{http://www.dtu.dk/english/Service/Phonebook/Person?id=59356&tab=1}
                   {Anand Natarajan, Ph.D},
              \href{http://www.dtu.dk/english/Service/Phonebook/Person?id=38753&tab=2&qt=dtupublicationquery}
                   {Nicolaos Antonion Cutululis, Ph.D},
              \href{http://www.dtu.dk/english/Service/Phonebook/Person?id=4583&tab=2&qt=dtupublicationquery}
                   {Kim Branner, Ph.D}, and
              \href{http://www.dtu.dk/Service/Telefonbog/Person?id=772&tab=2&qt=dtupublicationquery}
                   {John Michael Hansen, Ph.D}
        \end{innerlist}
%\end{outerlist}
\vspace{.1in}
\href{http://www.fresnostate.edu}{\textbf{California State University, Fresno}}, Fresno, CA,

%\begin{outerlist}
%\item[] M.S.,
        M.S., \href{http://www.fresnostate.edu/engineering/elec-computer/}
             {Electrical Engineering},
             Decemeber 2011
        \begin{innerlist}
        \item[] Topic: \emph{Efficient Drives for Single-phase AC Motors: Analysis and Applications}
        \item[] Advisor:
              \href{http://www.fresnostate.edu/engineering/elec-computer/faculty-staff.html}
                   {Nagy Bengiamin, Ph.D}
        \end{innerlist}
%\end{outerlist}
\vspace{.1in}
\href{http://www.mnsu.edu}{\textbf{Pontificia Universidad Javeriana}},
Santiago de Cali, Valle del Cauca, Colombia

        B.S., \href{http://www.cset.mnsu.edu/mathstat/}
             {Electronics Engineering}, October 2007

\section{Skills}
Programming languages/software

\begin{innerlist}
    \item[]  Matlab, Simulink, HAWC2, LabView, Python, git, Microsoft Office, C, Campbell Scientific instrumentation and \LaTeX.
    \end{innerlist}
Hardware
\begin{innerlist}
    \item[] Electric machines, power conversion, electric circuits, micro-controllers and DAQs.
    \end{innerlist}
Languages
\begin{innerlist}
 \item[] English -- Fluent.
 \item[] Danish -- Basic.
 \item[] Spanish -- Native speaker.
\end{innerlist}

\section{Refereed Journal Publications}
\vspace{-.1275in}
\begin{itemize}[leftmargin=*]
   \item[] {\bf Gallego-Calderon, J.}, and Natarajan A. (2015) ``\href{http://www.sciencedirect.com/science/article/pii/S0141029615005714}{Assessment of Wind Turbine Drive-train Fatigue Loads Under Torsional Excitation.}'' \textit{Engineering Structures}, 103, 189--202.
   \item[] {\bf J. Gallego-Calderon}, A. Natarajan and N. Dimitrov (2015). ``\href{http://ac.els-cdn.com/S187661021502175X/1-s2.0-S187661021502175X-main.pdf?_tid=3339a664-a58d-11e5-9c57-00000aab0f27&acdnat=1450446337_4374590c53b5cfdd00eb0da1e55d3a82}{Effects of bearing configuration in wind turbine gearbox reliability..}" \emph{Energy Procedia}, 80: 392--400.
   \item[] {\bf Gallego-Calderon, J.} and Bengiamin, N. (2013)``\href{http://www.ijme.us/issues/spring2013/abstracts/Z__IJME\%20spring\%202013\%20v13\%20n2\%20\%28paper\%203\%29.pdf}{Efficient Drives for
Single-phase AC Motors: Analysis and Applications.}" \emph{International Journal of Modern Engineering}, 13(2):25--33.
\end{itemize}

\section{Journal Papers in Preparation}
\vspace{-.1in}
%\begin{bibsection}
  \begin{itemize}[leftmargin=*]
   \item[] {\bf Gallego-Calderon, J.}, Natarajan A and  Cutululis, N.``Ultimate design load analysis of gearbox bearings under extreme loading." \emph{Under review by Wind Energy, September 2015.}
\end{itemize}


\section{Conference Papers}
\vspace{-.1in}
\begin{itemize}[leftmargin=*]
 \item []{\bf J. Gallego-Calderon}, K. Branner, A. Natarajan, N. Cutululis
and J. Hansen. ``\href{http://orbit.dtu.dk/fedora/objects/orbit:123931/datastreams/file_7a93ab44-e1b4-4a04-b9fa-b349801a6183/content}{Electromechanical Drivetrain Simulation.}'' \emph{9th PhD
Seminar on Wind Energy in Europe}, Gotland, Sweeden, 2013.
\end{itemize}

\halfblankline

\end{document}

%%%%%%%%%%%%%%%%%%%%%%%%%% End CV Document %%%%%%%%%%%%%%%%%%%%%%%%%%%%%

%----------------------------------------------------------------------%
% The following is copyright and licensing information for
% redistribution of this LaTeX source code; it also includes a liability
% statement. If this source code is not being redistributed to others,
% it may be omitted. It has no effect on the function of the above code.
%----------------------------------------------------------------------%
% Copyright (c) 2007, 2008, 2009, 2010, 2011 by Theodore P. Pavlic
%
% Unless otherwise expressly stated, this work is licensed under the
% Creative Commons Attribution-Noncommercial 3.0 United States License. To
% view a copy of this license, visit
% http://creativecommons.org/licenses/by-nc/3.0/us/ or send a letter to
% Creative Commons, 171 Second Street, Suite 300, San Francisco,
% California, 94105, USA.
%
% THE SOFTWARE IS PROVIDED "AS IS", WITHOUT WARRANTY OF ANY KIND, EXPRESS
% OR IMPLIED, INCLUDING BUT NOT LIMITED TO THE WARRANTIES OF
% MERCHANTABILITY, FITNESS FOR A PARTICULAR PURPOSE AND NONINFRINGEMENT.
% IN NO EVENT SHALL THE AUTHORS OR COPYRIGHT HOLDERS BE LIABLE FOR ANY
% CLAIM, DAMAGES OR OTHER LIABILITY, WHETHER IN AN ACTION OF CONTRACT,
% TORT OR OTHERWISE, ARISING FROM, OUT OF OR IN CONNECTION WITH THE
% SOFTWARE OR THE USE OR OTHER DEALINGS IN THE SOFTWARE.
%----------------------------------------------------------------------%
